\documentclass[12pt]{article}
\usepackage[spanish, english, es-tabla]{babel}
\usepackage[utf8]{inputenc}
\usepackage[left = 2cm, right = 2cm, bottom = 2cm, top = 3cm]{geometry}
\usepackage{amsmath, amssymb}
\usepackage{graphicx}
\usepackage [hidelinks]{hyperref}
\usepackage[usenames,dvipsnames]{xcolor}

%%SETUP FOR PDF PROPERTIES%%
\hypersetup{
	pdftitle={Hackathon AllWize: WizeTher},
	pdfsubject={Telecomunicaciones},
	pdfauthor={Enrique Fernández Sánchez, Lucía Francoso Fernández},
	pdfkeywords={Wize, AllWize, WizeTher, calidad aire, openData}
} 

%% SETUP FOR THE LINKS%%
\hypersetup{
	colorlinks,
    linkcolor=black,
	citecolor={blue!50!black},
	urlcolor=cyan
}

\begin{document}
	\selectlanguage{spanish}
	
	%centrar titulo y autores
	
	\title{\huge Hackathon AllWize \\ \LARGE \textit{WizeTher}\\}
	\author{\small Enrique Fernández Sánchez \\ \small Lucía Francoso Fernández}

	%quitar numeracion de la portada
	
	%% EDITAR PARA SEGUIMIENTO DE VERSIONES
	%\date{Revisión 19 Febrero 2021}
	
	\maketitle
	
	\pagebreak
	
	\tableofcontents
	
	\pagebreak
	
	% https://github.com/AllWize/allwize-training/tree/master/videos
	\section{Context. LPWAN}
	\noindent En el sentido de palabras topics, se mezcla en general lo que tiene que ver con Smar Citys, I4.0, IoT... Pero lo más importante es tener en claro lo que queremos solucionar. Es importante centrar el objetivo para solventar un problema. Necesidades/problemas --> soluciones específicas.
	
	\noindent IoT tiene origen en sensores meteorológicos, información remota, etc.
	
	\noindent LPWAN diseñadas para recurrir territorios muy grandes pero con bajo consumo. 
	
	\section{Retos}
	\begin{itemize}
		\item Security. Evitar MiM. Uso de TLS. 
		\item Autonomy. No cambiar baterías, o si las cambias que sea con un ciclo mayor. Búsqueda de duración de baterías de 10 años o que hagan harvesting de luz solar.
		\item Conectivity.
		\item Interoperability
		\item Awareness. Privacidad.
		\item Big data. Recopilación de datos. Solo recogemos la información necesaria.
		\item Business model. Monetización y sostenibilidad de la aplicación. \textbf{importante}
		\item Scalability. Escalabilidad de la solución. Optimiazción.
	\end{itemize}

	\section{Landscape}
	\noindent Importante ver las diferentes diferenciaciones en niveles (verticales y horizontales).
	
	% https://46eybw2v1nh52oe80d3bi91u-wpengine.netdna-ssl.com/wp-content/uploads/2016/03/Internet-of-Things-2016.png
	
	\pagebreak
	\section{Caja del hackathon.}
	\subsection{Placas controladoras.}
	\begin{itemize}
		\item AllWize K2. Arduino SAMD. 
		\item Carrier board para K2 (sin batería).
		\item AllWize K1. Wemos 8266 R1 D2 mini 
		\item x2 Antenas monopolo de cuarto de onda de 168MHz.
		\item IPEX a SMA
		\item x2 Cables micro USB
		\item Cables grove
	\end{itemize}
	\subsection{Sensores incluidos.}
	\begin{itemize}
		\item Multichannel Gas Sensor v2. I2C. 4 Variables.
		
		\item Barometer Sensor (BME280). Presión atmosferica, altitud, temperatura, humedad.
		
		\item Dust Sensor. Air quality
		
		\item Slide potenciometer 10k ohms
		
		\item RGB Led. WS2813 mini
		
		\item OLED display. 0.96"
		
		\item Hall sensor. Magnetico.
		
		\item Touch Sensor.
		
		\item Encoder.
		
		\item Loudness sensor.
		
		\item Air quality sensor.
		
		\item Sound sensor.
	\end{itemize}

	\section[Identificación de necesidades]{Identificación de necesidades}
	
	Una vez analizado el material que se nos ha proporcionado, y reflexionado acerca de la tecnología a emplear (en este caso, Wize), el siguiente paso fue hacer una investigación acerca del OpenData que ofrece el Ayuntamiento de Cartagena, así como de posibles líneas futuras que ha manifestado un interés en abarcar en cuanto al empleo de soluciones IoT con el objetivo de conseguir suficientes datos para toma de decisiones que conlleven una mejora en la calidad ambiental. \\
	
	\subsection{Calidad del aire}
	
	Así pues, encontramos un dashboard donde se reflejan los datos recopilados por diferentes estaciones, dentro de la zona del campo de Cartagena, relacionados con la calidad del aire. El link de acceso es el siguiente: 
	\href{https://www.cartagena.es/calidad_aire.asp}{Calidad del Aire (Campo de Cartagena)}. Dichas estaciones son, o se encuentran en: 
	
	\begin{itemize}
		\item Estación de Alumbres
		\item Estación de Escombreras
		\item Estación de la Aljorra
		\item Estación de Monpeán
	\end {itemize}
	
	Por las gráficas que se muestran, podemos observar que cada estación realiza varias mediciones al día de diferentes gases contaminantes; a la vez que dicha gráfica con datos variantes a lo largo del día (de ayer, hoy o previsión para mañana, según seleccionemos), se muestra una tabla, donde intuimos que se muestra el valor medio de concentración de dicho gas contaminante en el aire. Sin embargo, no existe (o al menos, no hemos encontrado) un portal con OpenData donde descargar en archivo csv/excel estos datos, ni un histórico con datos recopilados en días anteriores al de la consulta (como decimos, sólo existe un dashboard con datos del día anterior, actual y previsión para el siguiente). \\
	
	\subsection{Ruido}
	
	% https://www.cartagena.es/plantillas/14b.asp?pt_idpag=1431
		\href{https://www.cartagena.es/plantillas/14b.asp?pt_idpag=1431}{Portal de Transparencia del Ayto de Cartagena (Ciudad Sostenible)}
	
	Dentro del Portal de Transparencia del Ayto de Cartagena, en tanto a calidad ambiental, encontramos:
	
	\begin{itemize}
		\item Dentro de infraestructuras, agua potable, agua reciclada y reutilizada, e impactos ambientales
		\item Dentro de desarrollo sostenible, gases de efecto invernadero, niveles de calidad del aire (ya mencionado), medio natural, mapa de ruidos
	\end{itemize} 

Agua potable: un pdf de 2017 de una página
Agua reciclada y reutilizada: un pdf de 2017 de una página
Impactos ambientales: un pdf de una página "Iniciativas llevadas a cabo para mitigar los impactos ambientales de los productos y servicios"

Gases de efecto invernadero: un pdf de 104 paginas "inventario de emisiones y plan de accion para la energia sostenible del municipio de Cartagena"

Medio natural: %https://urbanismo.cartagena.es/medionatural/ no tiene na q ver, solo un pdf de 2010 sobre indicadores de sostenibilidad

Mapa de ruidos: % https://urbanismo.cartagena.es/PortalUrbanismo/Paginas/65 en urbanismo no tienen informes ambientales

		
		\subsection{Líneas futuras}
		
		El ayuntamiento de Cartagena ha mostrado interés, desde la Concejalía de Ciudad Sostenible y Proyectos Europeos, por la monitorización de la calidad del aire, ruido y aforos peatonales.
		
		%https://murciaplaza.com/cartagena-estudia-la-calidad-del-aire-ruidos-y-aforos-para-reducir-el-impacto-climatico-y-la-movilidad-urbana
		
	
\end{document}